The graph matching problem is among the most extensively researched topics in combinatorial optimization. Initial studies into the subject were motivated by practical issues including minimization of transportation costs~\cite{hitchcock1941distribution} or optimally assigning personnel to tasks~\cite{thorndike1950problem}. Over time, matching algorithms have found their use in scheduling, approximation algorithms and network switching among other problems. They play a crucial role in various other optimization algorithms including undirected shortest paths~\cite{lawler2001combinatorial}, planar max cut~\cite{hadlock1975finding}, metric traveling salesman~\cite{christofides2022worst}, and Chinese postman tours~\cite{edmonds1973matching}.

Graph matching algorithms are mainly divided into four groups based on the classes of graphs they operate on: whether they're weighted or unweighted and whether they're bipartite or non-bipartite. We will focus on the solutions to the case of weighted matching on general graphs.

To discuss further, we first define some terms. Consider a weighted graph $G = (V, E)$ with edge weight function $w$. We use $n = |V|$ and $m = |E|$ unless otherwise specified. For a graph $G'=(H, E')$ induced by a vertex set $H \subseteq V$ we denote $n_H = |H|$ and $m_H = |E'|$.

A \textit{matching} is a vertex-disjoint subset of the graph's edges. We call a matching \textit{perfect} if all vertices belong to exactly one edge of the matching. We define the \textit{weight} of a matching as the sum of its edges' weights and denote it as $w(M) = \sum_{e \in M} w(e)$.

In this work, we consider following variants of the maximum matching problem:

\begin{itemize}
    \item \textsc{Maximum Cardinality Matching} (\textsc{MCM}) Find a matching in a graph $G$ with maximum number of edges.
    \item \textsc{Maximum Weight Matching} (\textsc{MWM}) Find a matching in a weighted graph $G$ with maximum weight.
    \item \textsc{Maximum Weight Perfect Matching} (\textsc{MWPM}) Find a perfect matching in a weighted graph $G$ with maximum weight.
\end{itemize}

The \textsc{MWM} and \textsc{MWPM} are reducible to each other. For an instance $G=(V, E)$ of \textsc{MWM}, define a new graph $G'=(V', E')$ where $V'=V_1 \cup V_2$ and $V_1, V_2$ are two copies of $V$. The edges $E'$ consist of copies of edges in $E$ as well as one zero-weight edge between each corresponding pair of vertices in $V_1$ and $V_2$. A maximum weight perfect matching $M'$ on $G'$ can be used to obtain a maximum weight matching $M$ on $G$ by restricting the matching to only edges contained in $V_1$. If a vertex in $V_1$ is matched to its copy in $V_2$, it is unmatched in $M$. It's easy to see that $M$ is a maximum weight matching on $G$ as a matching with higher weight could be used to create a perfect matching on $G'$ with weight higher than $M'$. In the other direction, let a graph $G=(V, E)$ with weight function $w$ be an instance of \textsc{MWPM}. Construct a weight function $w'(e) = w(e) + nN$. A maximum weight matching on the graph $G' = G$ with weight function $w'$ must have the maximum possible number of edges as the $nN$ term in the definition $w'$ ensures that any matching with more edges has a higher weight.

The first polynomial algorithm for maximum weight matching in general graphs was given by Edmonds in~\cite{edmonds1965maximum}. It uses the primal-dual method and relies on his previous work on an algorithm for maximum cardinality matching~\cite{edmonds1965paths}. It is called the blossom algorithm after blossoms, which are certain odd length cycles, a feature that notably doesn't appear in bipartite graphs and is a major source of complexity in algorithms for general graphs. The running time given by Edmonds was $O(n^m)$. It was then independently improved to $O(n^3)$ by~\cite{lawler2001combinatorial} and~\cite{gabow1974implementation}. Many implementations of Edmonds' blossom algorithm followed with improved theoretical time complexity. The current best runs in $O(nm + n^2 \log n)$ due to~\cite{gabow1990data}, which is in some sense optimal. The blossom algorithm works in $O(n)$ phases, a single of which can be used to sort $n$ numbers and therefore requires $\Omega(m + n \log n)$ time in an appropriate model of computation. Other algorithms use an approach called scaling to solve the problem for graphs with integer edge weights. The weights are exposed one bit a time. In the $i$th scale, the algorithm computes the optimal solution for weights $w_i$ consisting of $i$ significant bits of $w$. The solution is then used to more efficiently solve the next scale. The first algorithm based on this approach was presented in~\cite{gabow1984efficient} and runs in time $O(n^{3/4}m)$. This was later improved to $O(m \sqrt{n\alpha(m, n) \log n} \log(nN))$ in~\cite{gabow1991faster} and lately $O(m \sqrt{n} \log(nN))$ in~\cite{duan2018scaling}. An algebraic randomized algorithm by~\cite{cygan2015algorithmic} runs in $O(n^\omega N)$ with high probability for the matrix multiplication exponent $\omega$.

A summary of existing algorithms is presented in table~\ref{tab:complexity}.

\begin{table}
\centering
\renewcommand{\arraystretch}{1.25}
\setlength{\tabcolsep}{1em}
\begin{tabular}{|l|l|l|}
\hline
Author & Year & Running time \\
\hline
Edmonds & 1965 & $n^2m$ \\
\hline
Gabow & 1974 & $n^3$ \\
\hline
Lawler & 1976 & $n^3$ \\
\hline
Gabow & 1985 & $n^{3/4}m \log N$ \\
\hline
Galil, Micali \& Gabow & 1986 & $nm \log n$ \\
\hline
Gabow, Galil \& Spencer & 1989 & $nm \log \log \log_{2 + m/2} + n^2\log n$ \\
\hline
Gabow & 1990 & $nm + n^2 \log n$ \\
\hline
Gabow \& Tarjan & 1991 & $m \sqrt{n \alpha(m, n) \log n} \log (nN)$ \\
\hline
Cygan, Gabow \& Sankowski & 2012 & $n^\omega N$ \\
\hline
Duan, Pettie \& Su & 2018 & $m \sqrt{n} \log(nN)$ \\
\hline
\end{tabular}
\caption{Summary of maximum weight matching algorithms}\label{tab:complexity}
\end{table}

We will take a closer look at a selection of maximum weight matching algorithms, namely Edmonds' original $O(n^2m)$ algorithm, Gabow's $O(n^3)$ algorithm, Galil, Micali \& Gabow's $O(nm \log n)$ algorithm and Gabow's first scaling algorithm running in $O(n^{3/4}m \log N)$. We implement all of them in C++20 as part of the open-source KOALA NetworKit library~\cite{koala-networkit}. In~\autoref{chap:algo} we describe how they work and how to implement them. In~\autoref{chap:results}, we present results of computational tests and compare the performance of our implementations.
