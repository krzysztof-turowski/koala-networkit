\section{Burer algorithm}
\label{sec:burer}

    \subsection{Idea and correctness}

The Burer algorithm introduced in \cite{burer2002rank} is a heuristic method designed to find an approximate solution to the max-cut problem.

\subsubsection{Formulation and Relaxation}

The Max-Cut problem can be formulated as the following binary quadratic program:
\[
\text{maximize} \quad \frac{1}{2} \sum_{i < j} w(i, j) \cdot (1 - x_i x_j) \quad \text{subject to} \quad x_i \in \{-1, 1\}, \text{ for all } i,
\]
where \( w(i, j)\) represents the weight of the edge between vertices \(i\) and \(j\), and \(x_i\) is a binary variable indicating the subset to which vertex \(i\) belongs.

This problem can be equivalently written as:
\[
\text{minimize} \quad \sum_{i < j} w(i, j) \cdot x_i x_j \quad \text{subject to} \quad x_i \in \{-1, 1\}, \text{ for all } i,
\]

Further, this can be relaxed into a semidefinite programming (SDP) problem:
\[
\text{minimize} \quad \frac{1}{2} W \cdot X \quad \text{subject to} \quad X_{ii} = 1 \text{ for } i = 1, \ldots, n, \quad X \succeq 0
\]
where \(W\) is the matrix of weights \( w(i, j)\), \(X\) is a positive semidefinite matrix, which means that it can be represented as \(X = V^T V\) for some matrix \(V\). This property ensures that all eigenvalues of \(X\) are non-negative and implies that \(X\) does not have any negative quadratic forms.


\subsubsection{ Rank-Two Relaxation }

Instead of solving the full SDP relaxation, Burer et al. propose a rank-two relaxation:
\[
\text{minimize} \quad f(\theta) = \frac{1}{2} W \cdot \cos(T(\theta)) \quad \text{where} \quad T_{ij}(\theta) = \theta_i - \theta_j
\]
In this formulation, each variable \(x_i\) is represented by an angle \(\theta_i\) on the unit circle. The matrix \(T(\theta)\) contains the pairwise differences of these angles.

Using polar coordinates, each \(x_i\) is mapped to \(\theta_i\) such that:
\[
v_i = \begin{pmatrix} \cos(\theta_i) \\ \sin(\theta_i) \end{pmatrix}
\]
and the inner product \(x_i x_j\) is represented as \(\cos(\theta_i - \theta_j)\).

\subsubsection{ Algorithmic Insight }

The core of the Burer algorithm involves the following steps:

\begin{enumerate}
    \item Start with an initial guess for \(\theta\).
    \item Use a gradient descent method to minimize the function \(f(\theta)\). The gradient \(\nabla f(\theta)\) is given by:
    \[
    \frac{\partial f(\theta)}{\partial \theta_j} = \sum_{k}  w(k, j) \cdot \sin(\theta_k - \theta_j).
    \]
    \item Once a local minimum \(\theta\) is found, generate a cut using the \texttt{procedureCut}. This involves partitioning the unit circle at various angles to find the best cut.
\end{enumerate}


\subsubsection{ ProcedureCut }
\label{burer:procedureCut}
The \texttt{procedureCut} is an efficient method to generate the best possible cut from a given angle configuration \(\theta\) (see Section 4 in \cite{burer2002rank}). Here’s how it works:

\begin{enumerate}
    \item Given \(\theta\), sort the angles \(\theta_i\) such that \(\theta_1 \leq \theta_2 \leq \ldots \leq \theta_n\).
    \item Initialize \(\alpha = 0\), \(\Gamma = -\infty\), and set \(i = 1\).
    \item Let \(j\) be the smallest index such that \(\theta_j > \pi\). If there is no such \(j\), set \(j = n + 1\).
    \item While \(\alpha \leq \pi\):
    \begin{enumerate}
        \item Generate a cut \(x\) by setting:
        \[
        x_i = \begin{cases} 
        +1 & \text{if } \theta_i \in [\alpha, \alpha + \pi) \\
        -1 & \text{otherwise}
        \end{cases}
        \]
        \item Compute the cut value \(\gamma(x)\).
        \item If \(\gamma(x) > \Gamma\), update \(\Gamma = \gamma(x)\) and \(x^* = x\).
        \item Adjust \(\alpha\):
        \[
        \alpha = \begin{cases} 
        \theta_i & \text{if } \theta_i \leq \theta_j - \pi \\
        \theta_j - \pi & \text{otherwise}
        \end{cases}
        \]
        \item Increment \(i\) or \(j\) accordingly - in the first case above increment \(j\), otherwise \(i\).
    \end{enumerate}
    \item Return the best cut \(x^*\).
\end{enumerate}.



\subsection{Implementation}

The \texttt{run} method orchestrates the entire process of finding the maximum cut. First we call \texttt{perturbTheta} which is responsible for distributing angle values uniformly and then performing gradient descent.

After that we calculate \texttt{procedureCut} which we discussed in \Cref{burer:procedureCut}, which calculates the minimum cut.

\begin{minted}[linenos]{cpp}
void RankTwoRelaxationMaxCut::run() {
    theta.resize(graph->numberOfNodes());
    distributeThetaEvenly();
    perturbTheta();
    maxCutSet = procedureCut();
    maxCutValue = calculateCutValue(maxCutSet);
    if (maxCutSet[graph->numberOfNodes() - 1] == true) {
        maxCutSet[graph->numberOfNodes() - 1] = false;
        double candidateValue = calculateCutValue(maxCutSet);
        if (candidateValue > maxCutValue) {
            maxCutValue = candidateValue;
        }
    }
}
\end{minted}

\texttt{procedureCut} implementation:

\begin{minted}[linenos]{cpp}
std::vector<bool> RankTwoRelaxationMaxCut::procedureCut() {
    double bestValue = -std::numeric_limits<double>::infinity();
    std::vector<bool> bestCut(graph->numberOfNodes()), x(graph->numberOfNodes());
    for (double alpha = 0; alpha <= M_PI; alpha += 0.01) {
        for (int i = 0; i < graph->numberOfNodes(); ++i) {
            x[i] = (theta[i] >= alpha && theta[i] < alpha + M_PI) ? true : false;
        }
        double value = calculateCutValue(x);
        if (value > bestValue) {
            bestValue = value;
            bestCut = x;
        }
    }
    return bestCut;
}
\end{minted}

\subsection{Time complexity}

Let \( T \) be the number of gradient descent iterations and \( k \) the number of partitions in \texttt{procedureCut}.

Combining all components, the overall time complexity of the Burer Algorithm can be summarized as follows:

\begin{itemize}
    \item \textbf{Initialization:} \(\mathcal{O}(|V|)\)
    \item \textbf{Gradient Descent:} \(\mathcal{O}(T \cdot |V|^2)\)
    \item \textbf{Cut Evaluation:} \(\mathcal{O}(|V|^2)\)
    \item \textbf{Procedure-CUT:} \(\mathcal{O}(k \cdot |V|^2)\), where \(k\) is the number of partitions.
\end{itemize}

Thus, the total time complexity of the Burer Algorithm is dominated by the gradient descent and \texttt{procedureCut} steps, yielding:
\[
\mathcal{O}(T \cdot |V|^2 + k \cdot |V|^2) = \mathcal{O}((T + k) \cdot |V|^2)
\]

In practical scenarios, \(T\) and \(k\) are constants or logarithmic relative to \(|V|\), making the algorithm efficient for large graphs.