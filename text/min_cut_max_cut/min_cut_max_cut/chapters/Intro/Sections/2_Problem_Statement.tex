\section{Problem statement}

This paper is dedicated to the in-depth investigation of two fundamental problems in graph theory: the max-cut and min-cut problems as formalized in \Cref{def:maxCut} and \Cref{def:minCut}. Throughout most of this paper, we will work with \emph{undirected graphs}, with a primary focus on searching for global min-cut and max-cut values. It is worth mentioning that only in the sections on Equivalence (\Cref{sec:mincut-maxflow}) and the Hao-Orlin algorithm (\Cref{sec:haoOrlin}), we will consider \emph{directed graphs}. Our main objective is to implement, analyze, and compare various algorithms designed to find optimal and approximate solutions to these problems, focusing on both their theoretical efficiency and practical applicability.

The techniques to be utilized in this research include the implementation of various algorithms—ranging from straightforward naive approaches and branch and bound strategies to more sophisticated techniques such as the Goemans-Williamson algorithm for max-cut and the Stoer-Wagner, Karger's, and Hao-Orlin algorithms for min-cut. We will analyze these algorithms for correctness, time complexity, and their practical utility across different graph structures characterized by varying size, density, and weight distributions.

Current algorithms allows min-cut can to be solved in polynomial time using among others the Ford-Fulkerson algorithm \cite{ford1956maximal}. In contrast, the max-cut problem is known to be NP-hard \cite{karp1972reducibility}, making it a more challenging problem from a computational complexity perspective.

To be more specific, the max-cut Problem is APX-hard, meaning that there is no polynomial-time approximation scheme (PTAS), arbitrarily close to the optimal solution, for it, unless P = NP \cite{papadimitriou1991optimization}. Thus, there is \( \epsilon > 0 \), such no \((1 + \epsilon)\)-approximation or \((1 - \epsilon)\)-approximation algorithm exists for max-cut. If we assume that the unique games conjecture (UGC) is correct, then the best known approximation ratio for max-cut would be \( 0.87856 - \epsilon \) for any \( \epsilon > 0 \) \cite{khot2007optimal}. However, if it is not true, then it is proven that this problem is NP-hard with approximation ratio better than \( \frac{16}{17} \) \cite{hastad2001optimal}. Thus, every known polynomial-time approximation algorithm achieves an approximation ratio strictly less than one.

In certain graph classes, the min-cut problem can be solved much faster. For example, in planar graphs, an optimal min-cut can be found in \(O(|V| \log^2 |V|)\) using algorithm by Łącki and Sankowski \cite{lacki2011min}. Similarly, the max-cut problem, while generally NP-hard, can be tractable in specific cases \cite{garey1979computers}. For instance, in planar graphs, the max-cut problem can be solved in polynomial time, specifically \( O(|V|^{1.5} \log V) \), by transforming the problem into a minimum-weight perfect matching problem on an associated graph \cite{liersch_pardella_2008}. In the context of parameterized complexity, the max-cut problem parameterized by the size of the cut \(k\) is fixed-parameter tractable (\textrm{FPT}) \cite{downey2013fundamentals}. This means that there exists an algorithm that solves the max-cut problem in \(f(k) \cdot |I|^{O(1)}\) time, where \(f\) is a function depending only on \(k\) and \(|I|\) is the size of the input. However, certain parameterizations of the max-cut problem, such as parameterizing by the number of vertices, lead to \textrm{W[1]}-hard problems, indicating that no FPT algorithm is likely to exist for these cases unless \textrm{FPT}=
\textrm{W[1]} \cite{downey2013fundamentals}.

Further, this paper will undertake a comparative evaluation based on experimental results to determine under what conditions each algorithm exhibits optimal performance. The duality between the max-flow and min-cut problems will also be explored to demonstrate how solutions to one can inform solutions to the other, enriching both the theoretical understanding and practical application perspectives.