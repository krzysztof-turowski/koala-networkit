\section{\textsc{Mis3}, \textsc{Mis4} and \textsc{Mis5}}

\subsection{Special case: graphs with $\Delta(G) \leq 2$}

We start by introducing a special case $\Delta(G) \leq 2$ and then use this algorithm in a general solution that improves running time. 

It is easy to observe that in graphs with a property $\Delta(G) \leq 2$ we have three types of structures.
\begin{itemize}
    \item isolated vertices with degree $0$,
    \item paths with lengths $\geq 2$
    \item cycles with lengths $\geq 3$
\end{itemize}

Each of these structures is a separate connected component, so we can solve the maximum independent set for them separately and then add results together.

\begin{defn}[component]
A \emph{component} of a graph $G$ is its maximal connected induced subgraph.
\end{defn}

\begin{algorithm}[H]
\caption{\textsc{PolyMis}}\label{alg:poly}
\begin{algorithmic}[1]
\Require a graph $G=(V,E)$ with $\Delta(G) \leq 2$
\Ensure the maximum independent set of $G$
\Procedure{PolyMis}{graph $G$}
    \State $I \gets \emptyset $
    \ForEach{$v\in V$}
        \State $visited[v] \gets false$
    \EndFor
    \ForEach{$v \in V$ \textbf{such that} $\deg(v)=0$}
        \State $I \gets I \cup \{v\}$
        \State $visited[v] \gets true$
    \EndFor
    \ForEach{$v \in V$ \textbf{such that} $\deg(v)=1 \land visited[v]=false$}
        \State $P \gets$ path containing $v$
        \State add $\lceil\|P|/2\rceil$ not neighboring vertices from $P$ to $I$
        \ForEach{$v \in P$}
            \State $visited[v] \gets true$
        \EndFor
    \EndFor
    \ForEach{$v \in V$ \textbf{such that} $visited[v]=false$}
        \State $C\gets$ cycle containing $v$
        \State add $\lfloor\|C|/2\rfloor$ not neighboring vertices from $C$ to $I$
        \ForEach{$v \in C$}
            \State $visited[v] \gets true$
        \EndFor
    \EndFor
    \State \Return $I$
\EndProcedure
\end{algorithmic}
\end{algorithm}

For the sake of clarity, the algorithm is presented in not the most concise, but still asymptotically optimal (in terms of $O(\cdot)$ notation).

\textsc{PolyMis} algorithm deals with these structures type by type and consecutively adds vertices to $I$. Vertices with degree $0$ can be simply added to $I$. For even and odd paths on $n$ vertices, it is easy to spot that the respective maximum independent sets would have $\lceil\frac{n}{2}\rceil$ vertices e.g. by taking every second vertex from one of its ends. Similarly, for cycles on $n$ vertices, one can find an independent set of size $\lfloor\frac{n}{2}\rfloor$.

Every part of the \textsc{PolyMis} algorithm works in a linear time regarding the number of vertices of $G$. It, of course, means that the whole algorithm has polynomial complexity.

Using this algorithm as a subprocedure for solving the general case allows us to branch only on vertices with higher degrees. 

\subsection{General case algorithms}

We proceed with three quite similar algorithms. We will discuss their correctness and complexity collectively.

\begin{algorithm}[H]
\caption{\textsc{Mis3}}\label{mis3}
\begin{algorithmic}[1]
\Require a graph $G=(V,E)$
\Ensure the maximum independent set of $G$
\Procedure{Mis3}{graph $G$}
    \If{$\exists v \in V \colon \deg (v) = 0$} 
         \State \Return $\Call{Mis3}{G\setminus \{v\}} \cup \{v\}$
    \EndIf
    \If{$\exists v \in V \colon \deg (v) = 1$} 
        \State \Return $\Call{Mis3}{G\setminus N[v]} \cup \{v\}$
    \EndIf
    \If{$\Delta(G) \geq 3$} 
        \State choose arbitrary vertex $v$ of maximum degree in $G$
        \State $A \gets \Call{Mis3}{G\setminus N[v]} \cup \{v\}$
        \State $B \gets \Call{Mis3}{G\setminus \{v\}}$
        \State \Return largest of the sets: $A, B$
    \Else
        \State \Return maximum independent set using \Cref{alg:poly}
    \EndIf
\EndProcedure
\end{algorithmic}
\end{algorithm}

\begin{algorithm}[H]
\caption{\textsc{Mis4}}\label{alg:mis4}
\begin{algorithmic}[1]
\Require a graph $G=(V,E)$
\Ensure the maximum independent set of $G$
\Procedure{Mis4}{graph $G$}
    \If{$\Delta(G) \geq 3$} 
        \State choose arbitrary vertex $v$ of degree $\deg(v) \geq 3$ in $G$
        \State $A \gets \Call{Mis4}{G\setminus N[v]} \cup \{v\}$
        \State $B \gets \Call{Mis4}{G\setminus \{v\}}$
        \State \Return largest of the sets: $A, B$
    \Else 
        \State \Return maximum independent set using \Cref{alg:poly}
    \EndIf
\EndProcedure
\end{algorithmic}
\end{algorithm}

\begin{algorithm}[H]
\caption{\textsc{Mis5}}\label{alg:mis5}
\begin{algorithmic}[1]
\Require a graph $G=(V,E)$
\Ensure the maximum independent set of $G$
\Procedure{Mis5}{graph $G$}
    \If{$\Delta(G) \geq 3$}
        \State choose arbitrary vertex $v$ of maximum degree in $G$
        \State $A \gets \Call{Mis5}{G\setminus N[v]} \cup \{v\}$
        \State $B \gets \Call{Mis5}{G\setminus \{v\}}$
        \State \Return the largest of the sets: $A, B$
    \Else
        \State \Return maximum independent set using \Cref{alg:poly}
    \EndIf
\EndProcedure
\end{algorithmic}
\end{algorithm}

\subsection{Correctness}

Let's start with the introduction of \emph{standard branching}. This is the default branching method that will be used throughout all the following algorithms. Later on, we will refine it and introduce mirror branching.
\begin{lemma}
Let $\alpha$ be an algorithm finding the maximum independent set. Then:
$$
\alpha(G) = max\{\alpha(G\setminus \{v\}),1+\alpha(G\setminus N[v])\}
$$
\end{lemma}
It translates into "An algorithm will find the maximum independent set by either discarding $v$ from $G$ or selecting $v$ to the maximum independent set and discarding $N[v]$ from the independent set". $N[v]$ vertices can no longer be a part of a solution after adding $v$ to the independent set. There are no other options of choice, so at least in one of them, an algorithm will find the maximum independent set. 

Standard branching can be applied on any vertex in graph $G$ and the correctness will hold. In the case of \textsc{Mis3}, \textsc{Mis4} and \textsc{Mis5} an algorithm selects a vertex with a degree of at least $3$ be it either the smallest or largest degree depending on the algorithm. Algorithms also save both results for selecting and not selecting $v$ to $A$ and $B$ and then select the set with the most elements. 

These three algorithms also slightly differ in the way of dealing with small degree vertices. \textsc{Mis4} and \textsc{Mis5} do not do anything special and eventually solve graphs with $\Delta \leq 2$ with a polynomial algorithm. \textsc{Mis3} is somewhat unique because it deals with vertices of degree $0$ and $1$ right away. \textsc{Mis3} adds vertices of degree $0$ and $1$ to the independent set. Vertices of degree $0$ trivially must be added to the maximum independent set. Selecting vertices of degree $1$ is also always optimal, as selecting its neighbor -- $N(v)$ would lead to the removal of at least $v, N(v)$ and possibly another vertex connected to $N(v)$. 

Ultimately, we are left with a graph with a property $\Delta \leq 2$. The polynomial algorithm \cite{alg:poly} finishes the job.

\subsection{Computational complexity in \textsc{Mis3}, \textsc{Mis4} and \textsc{Mis5}}

Now, after this introduction, let us try to bound \textsc{Mis3}, \textsc{Mis4} and \textsc{Mis5} with that method. Algorithms has only one branching rule when following Mis$(G\setminus N[v])$ or Mis$(G\setminus \{v\})$ down the tree. For vertex $v$ this reduces the size of the tree accordingly by $d(v) + 1$ and $1$. This implies the recurrence:

$$
T(n) \leq T(n-(d(v)+1)) + T(n - 1)
$$

The worst case scenario is going to be when $\deg(v)=3$, so we get
$$
T(n) \leq T(n - 4) + T(n - 1)
$$

Instead of solving it like for \textsc{Mis1} algorithm, we can use the branching vectors method to bound complexity. The branching vector for these three algorithms is $(4,1)$, which can be computed to $\tau(4,1)=1.3803$. \textsc{Mis3}, \textsc{Mis4}, \textsc{Mis5} all are bounded by $O^*(1.13803^n)$. We will see how they to each other and \textsc{Mis1} in \Cref{chp:benchmark}.
