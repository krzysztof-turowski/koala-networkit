The algorithms were implemented in C++. \textit{NetworKit} and \textit{Boost} frameworks were used in for data structures (e.g. graph) and algorithms (e.g. for random graph generation and finding maximum matching). The implementation's source code was added to the \textbf{koala-networkit} repository and is available at \href{https://github.com/krzysztof-turowski/koala-networkit/pull/8}{\texttt{https://github.com/krzysztof-turowski/koala-networkit/pull/8}}. Classes BranchAndReduceMDS$\langle$GrandoniMSC$\rangle$, BranchAndReduceMDS$\langle$FominGrandoniKratschMSC$\rangle$, BranchAndReduceMDS$\langle$RooijBodlaenderMSC$\rangle$, FominKratschWoegingerMDS, SchiermeyerMDS implement algorithms of GRANDONI, Fomin-Grandoni-Kratsch, van Rooij-Bodlaender, Fomin-Kratsch-Woeginger, Schiermeyer, respectively.
\section{Implementation decisions}
\subsection{Measure and conquer algorithms}
\begin{enumerate}
    \item Due to the fact that implementation is expected to run on instances which contain several dozen of vertices, the asymptotic time execution is not the only factor one should take care of. To enhance implementation's running time one may consider avoiding some operations which are slow in practice. For instance copying and allocating memory for the structures defining the correspondence of elements belonging to the sets. The alternative is modifying the structures before each call to the sub-problem and retrieving them whenever it returns.
    \item Family of sets in the sub-problems do not contain empty sets. These however may be present in memory representations of structures used. They are not counted to solution cardinality and do not influence any high-level logic of the algorithm. 
\end{enumerate}
\subsection{Algorithms for minimum optional dominating set}
\begin{enumerate}
    \item Similarly as for measure and conquer copying structures used in recursive calls in avoided, instead these are modified and then restored properly after recursive call.
    \item In the parts of the algorithms, where small dominating set candidates are tested (of cardinality up to $\frac{1}{3}|V(G)|$ and $\frac{3}{8}|V(G)|$ in algorithms of Schiermeyer, Fomin-Kratsch-Woeginger, respectively), search is conducted in the ascending cardinality order.
\end{enumerate}

\section{Data sets}
Considered graphs were: 
\begin{itemize}
    \item all (non-isomorphic) graphs of cardinality up to 10. These graphs were sourced from a site \url{https://users.cecs.anu.edu.au/~bdm/data/graphs.html}.
    \item because number of different graphs grows too quickly, bigger graphs considered were generated randomly using \textit{NetworKit}, with each pair of vertices was considered independently, with equal probability of being connected by an edge. 
\end{itemize}
\subsection{Correctness}
For each implemented algorithm the implementation correctness was tested against exhaustive algorithm on all small graphs. Moreover implemented algorithms were checked against themselves on bigger random graphs.

\section{Experiments}
Considered graph groups were of average degree 3, 6 and $(|V(G)|-1)/2$. For each of them and each integer $n = |V(G)|$ from 11 to 64, 100 random graphs were created. Each algorithm was executed on the same set of graphs. The average and worst (among these 100 graphs) execution times are presented in the following figures.
\begin{itemize}
    \item graphs with average degree 3 can be solved very fast, especially with branch and reduce algorithms (although when $n > 32$, the execution time experiences high variance (\Cref{fig:l3_worst}). It is due to the fact, that many reduction rules are focused on sets of cardinality 2. Fomin-Kratsch-Woeginger algorithm also solves these type of graphs quickly, because it the beginning it branches of vertices of degree one and two. Algorithm of Schiermeyer is the slowest one. In such graphs dominating sets are relatively large and it cannot find one with cardinality $\leq \frac{n}{3}$, thus performing worse.
    \item small graphs with average degree 6 can be solved quicker with Schiermeyer and Fomin-Kratsch-Woeginger algorithms. They are quite dense in this case. However for larger (quite sparse) graphs branch and reduce algorithms outperform them.
    \item algorithms of \citeauthor{GRANDONI2006209}, Fomin-Grandoni-Kratsch, van Rooij-Bodlaender have similar time execution. In fact this is understandable: in \cite{VANROOIJ20112147} it was proved that the running times are $O(1.5709^n)$, $O(1.5169^n)$, $O(1.4969^n)$, respectively. It is unknown, whether this bounds are tight. It is possible that asymptotically these differ more or less that it follows from the measure and conquer analysis. It is also possible that the hardest graphs for these algorithms are completely different from the ones considered. 
    \item when graphs are very dense (average degree $\approx(n-1)/2$), algorithms of Schiermeyer and Fomin-Kratsch-Woeginger seem to be better. It is due to the fact, that minimum dominating sets are usually small in such scenario, possibly $O(\log(n))$.
    \item in the \Cref{fig:small} small graphs were benchmark-ed. All algorithms visibly outperform exhaustive algorithm for input graphs with just 10 vertices. 
\end{itemize}

\begin{figure}[H]
    \centering
    \includegraphics[width=0.9\textwidth]{figures/m3_average.png}
    \caption{average degree = 3}
    \label{fig:m3_average}
\end{figure}

\begin{figure}[H]
    \centering
    \includegraphics[width=0.9\textwidth]{figures/m3_worst.png}
    \caption{average degree = 3}
    \label{fig:m3_worst}
\end{figure}

\begin{figure}[H]
    \centering
    \includegraphics[width=0.9\textwidth]{figures/l3_average.png}
    \caption{average degree = 3}
    \label{fig:l3_average}
\end{figure}

\begin{figure}[H]
    \centering
    \includegraphics[width=0.9\textwidth]{figures/l3_worst.png}
    \caption{average degree = 3}
    \label{fig:l3_worst}
\end{figure}

\begin{figure}[H]
    \centering
    \includegraphics[width=0.9\textwidth]{figures/m6_average.png}
    \caption{average degree = 6}
    \label{fig:m6_average}
\end{figure}

\begin{figure}[H]
    \centering
    \includegraphics[width=0.9\textwidth]{figures/m6_worst.png}
    \caption{average degree = 6}
    \label{fig:m6_worst}
\end{figure}

\begin{figure}[H]
    \centering
    \includegraphics[width=0.9\textwidth]{figures/l6_average.png}
    \caption{average degree = 6}
    \label{fig:l6_average}
\end{figure}

\begin{figure}[H]
    \centering
    \includegraphics[width=0.9\textwidth]{figures/l6_worst.png}
    \caption{average degree = 6}
    \label{fig:l6_worst}
\end{figure}

\begin{figure}[H]
    \centering
    \includegraphics[width=0.9\textwidth]{figures/dense_average.png}
    \caption{average degree = $\frac{n-1}{2}$}
    \label{fig:dense_average}
\end{figure}

\begin{figure}[H]
    \centering
    \includegraphics[width=0.9\textwidth]{figures/dense_worst.png}
    \caption{average degree = $\frac{n-1}{2}$}
    \label{fig:dense_worst}
\end{figure}

\begin{figure}[H]
    \centering
    \includegraphics[width=0.9\textwidth]{figures/small_all_average.png}
    \caption{all graphs between 4 and 10 vertices, average time}
    \label{fig:small}
\end{figure}